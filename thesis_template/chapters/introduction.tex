\chapter{Introduction}\label{chap:introduction}


Motivate your research and outline the research gap in this chapter. Why is your thesis relevant and what do you address, what has not been addressed before. 


Machine learning which is a domain of Artificial Intelligence has come across many developments during the recent years, with the increase in availability of numerous data from different domains such as Medicine, Astronomy, Housing, Wearable Devices, Manufacturing, E-commerce and much more. Several Applications of machine learning include Object recognition, Natural language processing, Speech and handwriting recognition, Recommendation systems and many more. ML nowadays is employed in most of the digital systems, for example the advertisements we come across in our social media pages, product recommendations in online shops based on our purchase history. These systems make use of the previous data which has been generated from our online activity and they build a model out of the data, this model tries to predict our next outcome based on the previous observations which we had made. as for advertising and product recommendations several other domains of data such as wearable devices like smart watches, fitness trackers these devices record your daily walking distances, biking, places you visit. Based on these data the system would assist you in your daily activities and also provide health based recommendations.


Many of the industries in these domains or several other such entities are digitizing there processes in a large scale. with recent 


are aiding the progress of machine learning techniques and developing new algorithms to further improve the learning process. The process of machine learning is divided into certain categories such as Supervised, Semi Supervised and unsupervised learning and several others namely Active and reinforcement learning. In order to decide which technique to use we need to first look at the data at hand. As in any machine learning project or task one has to gather the required data at first, data is the key point in describing what type of task does this belong to, for example a data-set consisting of samples of different fruit properties like color, PH,size, season and these properties or features are labelled to there respective fruit name like Apple or orange. Here this is a classification task where the machine learning model will learn from the samples and classify the fruit to which it belongs.  

\cite{6147691}

Machine learning is exponentially growing at the moment with big tech giants using machine learning to there everyday operations and processes. Large firms like Google and Microsoft have invested heavily in the research towards machine learning and providing machine learning as a service to the public. Due to advances in the domain of machine learning even a beginner in the field can easily build a model i.e the ease of usage with machine learning libraries has come to a great advantage in reducing production and execution time for constructing a model out of the data. Rapid exploration towards data-science has given a boost to the community in developing simpler and efficient ML libraries for use. These libraries are updated frequently with new algorithms, preprocessing, feature extraction methods as well as enabling the use of these methods in a simpler and in a easy to code manner. still in order to build a efficient production ready models out of available data one has to spend a lot of time analyzing the features of the data and evaluating the combination of parameters and hyper-parameters for the particular model which he/she is trying to build.

General Requirements to the thesis:

\begin{itemize}
	\item 60 pages of content in this format. Content does not include table of content, lists, appendices etc.
	\item Proper scientific referencing
	\item Introduction and Background should be less than 50\% of the thesis
	\item Images should be readable and in the proper size. 
\end{itemize}


\section{Research Questions}

Write down and explain your research questions (2-5)

\section{Structure of the Thesis}

Explain the structure of the thesis. 

\section{Example citation \& symbol reference}\label{sec:citation}
For symbols look at \cite{latex_symbols_2017}.


\section{Example reference}
Example reference: Look at chapter~\ref{chap:introduction}, for sections, look at section~\ref{sec:citation}.

\section{Example image}

\begin{figure}
	\centering
	\includegraphics[width=0.5\linewidth]{uni-logo}
	\caption{Meaningful caption for this image}
	\label{fig:uniLogo}
\end{figure}

Example figure reference: Look at Figure~\ref{fig:uniLogo} to see an image. It can be \texttt{jpg}, \texttt{png}, or best: \texttt{pdf} (if vector graphic).

\section{Example table}

\begin{table}
	\centering
	\begin{tabular}{lr}
		First column & Number column \\
		\hline
		Accuracy & 0.532 \\
		F1 score & 0.87
	\end{tabular}
	\caption{Meaningful caption for this table}
	\label{tab:result}
\end{table}

Table~\ref{tab:result} shows a simple table\footnote{Check \url{https://en.wikibooks.org/wiki/LaTeX/Tables} on syntax}